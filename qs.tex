\documentclass[a4paper,10pt]{article}
\usepackage{graphicx}
%\usepackage[active]{srcltx}
\usepackage{amsmath,amssymb} 
\usepackage{enumerate}


\def\noi{\noindent}%  
\def\CC{\mathbb{C}}%  
\def\PP{\mathbb{P}}%  
\def\NN{\mathbb{N}}%  
\def\RR{\mathbb{R}}% 
\def\II{\mathbb{I}}
\def\ZZ{\mathbb{Z}}% 
\def\QQ{\mathbb{Q}}% 
\def\cC{{\Cal C}}%  
\def\cF{{\Cal F}}%  
\def\cM{{\Cal M}}%  
\def\cS{{\Cal S}}%  
\def\cT{{\Cal T}}%  
\def\eps{\varepsilon}%  
\def\op{\operatorname}

\def\i{\subset}
\def\l{\!\left}
\def\r{\right}
\def\<{\langle}
\def\>{\rangle}
\def\:{\colon}
\def\abs#1{|#1|}

\begin{document}

\begin{center}
Quiz 1.
\end{center}


\ 

\noi{\bf 1.} 
Let $x, y \in F$, where $F$ is an ordered field.  
Suppose $0 < x < y$.  
Show that $$x^2 < y^2$$

\ 

\noi{\bf Solution:}
Since $x,y>0$, multiplying $x < y$ by $x$ and $y$, we get
\[x^2<yx\quad\text{and}\quad xy<y^2.\]
Therefore 
\[x^2<y^2.\]

\

\noi{\bf 2.} Show that the following identity holds in any field
$$(-x)y=-xy.$$

\ 

\noi{\bf Solution:} In addition to the properties of the field (Definition 1.1.5) we will use the identity $0y=0$ which was proved already.

\[xy+(-xy)=0=0y=(x+(-x))y=xy+(-x)y.\]
Therefore 
\[xy+(-xy)=xy+(-x)y.\]
Adding $(-xy)$ to both sides, we get
\[(-xy)=(-x)y.\]

\begin{center}
Quiz 2.
\end{center}


\ 

\noi{\bf 1.} Let $S \subset \RR$ be a nonempty set, bounded from above. 
Show that for every $\eps > 0$ there exists $x \in  S$ such that 
$$\sup S - \eps < x \le  \sup S.$$

\ 

\noi{\bf Solution:} The second inequality holds for any $x\in S$  since $\sup S$ is an upper bound of $S$.

If there is no $x\in S$ such that $\sup S - \eps < x$ then $x\le \sup S - \eps$ for any $x\in S$.
That is, $\sup S - \eps$ is an upper bound of $S$.
In particular $\sup S$ is not the least upper bound, a contradiction.

\ 

\noi{\bf 2.} Let $A,B\subset \RR$ be bounded nonempty sets.
Assume for any $a\in A$ there is $b\in B$ such that $a\le b$.
Show that $\sup A\le \sup B$.

\ 

\noi{\bf Solution:} Since for any $a\in A$ there is $b\in B$ such that $a\le b$,
any upper bound for $B$ is an upper bound for $A$.
In particular, $\sup B$ is an upper bound for $A$.
Therefore $\sup A\le \sup B$.

\begin{center}
Quiz 3.%NUMBER
\end{center}


\ 

\noi{\bf 1.} 
Let $A$ and $B$ be two nonempty bounded sets of real numbers.
Let
$$C := \{ a+b : a \in A, b \in B \}.$$
Show that $C$ is a bounded set and that
$$\sup\,C = \sup\,A + \sup\,B.$$

\ 

\noi{\bf Solution:} Since $a\le \sup A$ for any $a\in A$
and $b\le \sup B$ for any $b\in B$, we have
\[a+b\le \sup\,A + \sup\,B\]
for any $a\le \sup A$ and $b\in B$.
That is $\sup\,A + \sup\,B$ is an upper bound for $C$.

Note that for any $\eps>0$ there is $a\in A$ such that $a>\sup A-\tfrac\eps2$
and $b\in B$ such that $b>\sup B-\tfrac\eps2$.
Therefore $a+b>\sup A+\sup B-\eps$.
That is $\sup A+\sup B-\eps$ is not an upper bound for any $\eps>0$;
hence the statement follows.


\ 

\noi{\bf 2.} Give a definition of absolute value.

\

\noi{\bf Solution:}
\[|x|=\left[\begin{aligned}
             x&&\text{if $x\ge 0$;}
             \\
             -x&&\text{if $x< 0$.}
            \end{aligned}
 \right.
 \]

\ 

\noi Use it to prove that
$|-x|=|x|$
for any $x\in\RR$.

\

\noi{\bf Solution:} 
\begin{align*}
&\text{If $x=0$ then $x=-x$}&&&& \Rightarrow &&&|x|&=|-x|;
\\
&\text{If $x>0$ then $-x<0$}
&&\Rightarrow 
& |x|&=x\ \text{and}\ |-x|=x 
&&\Rightarrow& |x|&=|-x|;
\\
&\text{If $x<0$ then $-x>0$}
&&\Rightarrow &
|x|&=-x\ \text{and}\ |x|=-x
&&
\Rightarrow&
 |x|&=|-x|.
\end{align*}



\begin{center}
Quiz 4.
\end{center}



\ 

\noi{\bf 1.} 
Suppose $S$ is a set of disjoint open intervals in $\RR$.
That is, 
if $(a,b) \in S$ and $(c,d) \in S$, then either $(a,b) = (c,d)$
or $(a,b) \cap (c,d) = \emptyset$.

Prove $S$ is a countable set.

\

\noi{\bf Solution:} 
Since the set of rationals is dense in  $\RR$,
we can choose a rational number $q\in\QQ$ in each interval from $S$;
that is, there is an bijection from $S$ to a subset of $\QQ$.

Since $\QQ$ is countable, the statement follows.

\

\noi{\bf 2.} Show that the set of irrational numbers is uncountable.

\

\noi{\bf Solution:} Arguing by contradiction, assume the set irrational numbers $\II=\RR\backslash \QQ$ is countable.
In this case the $\RR$ can be presented as a union of two countable sets $\II$ and $\QQ$.
Therefore $\RR$ is countable.
The latter contradicts Cantor's theorem.


\begin{center}
Quiz 5.%NUMBER
\end{center}


\ 

\noi{\bf 1.} Let $\{ x_n \}$ be a sequence.
\begin{enumerate}[a)]
\item Show that $\lim\, x_n = 0$ (that is, the limit exists and is zero)
if and only if $\lim \abs{x_n} = 0$.

\

\noi{\bf Solution:} $\lim\, x_n = 0$ $\Leftrightarrow$ ``for any $\eps>0$ there is $M\in\NN$ such that 
$|x_n-0|<\eps$ for any $n\ge M$''.
Since
\[||x_n|-0|=|x_n|=|x_n-0|.\]
This statement is equivalent to the following
``for any $\eps>0$ there is $M\in\NN$ such that 
$||x_n|-0|<\eps$ for any $n\ge M$''.
The latter means that $|x_n|\to 0$.


\ 

\item Find an example such that $\{ \abs{x_n} \}$ converges and $\{ x_n \}$
diverges.

\

\noi{\bf Solution:} $x_n=(-1)^n$.

\end{enumerate}



\ 

\noi{\bf 2.} Prove that any convergent sequence has a unique limit.

\

See Proposition 2.1.6.

\end{document}